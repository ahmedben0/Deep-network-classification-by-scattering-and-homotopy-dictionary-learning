%%%%%%%%%%%%%%%%%%%%%%%%%%%%%%%%%%%%%%%%%
% Beamer Presentation
% LaTeX Template
% Version 1.0 (10/11/12)
%
% This template has been downloaded from:
% http://www.LaTeXTemplates.com
%
% License:
% CC BY-NC-SA 3.0 (http://creativecommons.org/licenses/by-nc-sa/3.0/)
%
%%%%%%%%%%%%%%%%%%%%%%%%%%%%%%%%%%%%%%%%%

%----------------------------------------------------------------------------------------
%	PACKAGES AND THEMES
%----------------------------------------------------------------------------------------

\documentclass{beamer}

\mode<presentation> {

% The Beamer class comes with a number of default slide themes
% which change the colors and layouts of slides. Below this is a list
% of all the themes, uncomment each in turn to see what they look like.

%\usetheme{default}
%\usetheme{AnnArbor}
%\usetheme{Antibes}
%\usetheme{Bergen}
%\usetheme{Berkeley}
\usetheme{Berlin}
%%\usetheme{Boadilla}
%\usetheme{CambridgeUS}
%%\usetheme{Copenhagen}
%\usetheme{Darmstadt}
%%\usetheme{Dresden}
%\usetheme{Frankfurt}
%\usetheme{Goettingen}
%\usetheme{Hannover}
%\usetheme{Ilmenau}
%\usetheme{JuanLesPins}
%\usetheme{Luebeck}
%%\usetheme{Madrid}
%\usetheme{Malmoe}
%\usetheme{Marburg}
%\usetheme{Montpellier}
%\usetheme{PaloAlto}
%\usetheme{Pittsburgh}
%\usetheme{Rochester}
%\usetheme{Singapore}
%\usetheme{Szeged}
%\usetheme{Warsaw}

% As well as themes, the Beamer class has a number of color themes
% for any slide theme. Uncomment each of these in turn to see how it
% changes the colors of your current slide theme.

%\usecolortheme{albatross}
%\usecolortheme{beaver}
%\usecolortheme{beetle}
%\usecolortheme{crane}
%\usecolortheme{dolphin}
%\usecolortheme{dove}
%\usecolortheme{fly}
%\usecolortheme{lily}
%\usecolortheme{orchid}
%\usecolortheme{rose}
%\usecolortheme{seagull}
%\usecolortheme{seahorse}
%\usecolortheme{whale}
%\usecolortheme{wolverine}

%\setbeamertemplate{footline} % To remove the footer line in all slides uncomment this line
\setbeamertemplate{footline}[page number] % To replace the footer line in all slides with a simple slide count uncomment this line

\setbeamertemplate{navigation symbols}{} % To remove the navigation symbols from the bottom of all slides uncomment this line
}

\usepackage{graphicx} % Allows including images
\usepackage{booktabs} % Allows the use of \toprule, \midrule and \bottomrule in tables

%----------------------------------------------------------------------------------------
%	TITLE PAGE
%----------------------------------------------------------------------------------------

\title[Scattering]{Deep network classification by scattering and homotopy dictionary learning} % The short title appears at the bottom of every slide, the full title is only on the title page

\author{Ahmed Ben Aissa - Elie Mokbel - Mohammed Fellaji} % Your name
\institute[CentraleSupélec] % Your institution as it will appear on the bottom of every slide, may be shorthand to save space
{
Ecole CentraleSupélec \\ % Your institution for the title page
\medskip

}
\date{\today} % Date, can be changed to a custom date

\begin{document}

\begin{frame}
\titlepage % Print the title page as the first slide
\end{frame}


\begin{frame}
\frametitle{Overview} % Table of contents slide, comment this block out to remove it
\tableofcontents % Throughout your presentation, if you choose to use \section{} and \subsection{} commands, these will automatically be printed on this slide as an overview of your presentation
\end{frame}

%----------------------------------------------------------------------------------------
%	PRESENTATION SLIDES
%----------------------------------------------------------------------------------------
%----------------------------------------------------------------------------------------

%------------------------------------------------
\section{18-11-2020} 
%------------------------------------------------% !TEX encoding = UTF-8 Unicode
\documentclass{article}
\usepackage[utf8]{inputenc}
\usepackage{geometry}
\geometry{letterpaper}
\usepackage[parfill]{parskip}
\usepackage{graphicx}

\usepackage[numbers,sort&compress]{natbib}
\usepackage{amssymb}
\usepackage{amsmath}
\usepackage[english]{babel}   %    
\usepackage[T1]{fontenc}
\usepackage[autolanguage]{numprint}
\usepackage{color}


\usepackage{hyperref}
\usepackage{listings}

\usepackage{tabto}
\usepackage{array}
\usepackage{amsmath}

\usepackage{hyperref}

\usepackage{url}
\usepackage{wrapfig}

\usepackage{amssymb}

\usepackage{caption}

\definecolor{backcolour}{rgb}{0.97,0.95,0.93}

\lstdefinestyle{mystyle}{
    backgroundcolor=\color{backcolour},
}

\author{\Large \textsc{\href{mailto:ahmed.benaissa@supelec.fr}{Ahmed Ben Aissa} - \href{mailto:elie.mokbel@supelec.fr}{Elie Mokbel} - \href{mailto:mohammed.fellaji@supelec.fr}{Mohammed Fellaji}}}
\date{\today}

\begin{document}

\hypersetup{pdfborder=0 0 0} 		%pour enlever le cadre rouge dans la tables des matières


\makeatletter
  \begin{titlepage}
  \centering
     {\large \textsc{   }}\\
    \centering
      \includegraphics[width=1 \textwidth]{figures/logo.png} \\
    \vspace{3cm}
      {\LARGE\textbf{Deep network classification by scattering and homotopy dictionary learning}\\
    \vspace{3cm}
    \centering
     {\Large \textsc{Supervisors : }}\\  
      {\href{mailto:michel.barret@centralesupelec.fr}{Michel Barret}  - \href{mailto:matthieu.bloch@ece.gatech.edu}{Matthieu Bloch} - \href{mailto:damien.rontani@centralesupelec.fr}{Damien Rontani} } \\
     \vspace{2em}
     	{\large \textsc{Authors : }}\\
        {\Large \@author} \\
        \vspace{4em}
        {\Large \@date} }\\
  \end{titlepage}
 
 
\makeatother

\tableofcontents
%% \listoffigures

%% \newpage
%% \listoftables



%%%%%%%%%%%%%%%

\newpage
\bibliographystyle{ieeetr} % plain : no order -- ieeetr : sorted
\nocite{*}    % print all references
\bibliography{reference/ref}

\end{document}


\subsection{Setting Up} 
%------------------------------------------------

%------------------------------------------------
\begin{frame}
\begin{itemize}
	\item  \href{https://github.com/fellajimed/Deep-network-classification-by-scattering-and-homotopy-dictionary-learning}{public github repo}
	
	
	\begin{itemize}
    	\item papers;
    	\item presentations 
    	\item report pdf
    \end{itemize}

\end{itemize}
\end{frame}
%------------------------------------------------

\subsection{introduction : Invariant Scattering Convolution Networks}

%------------------------------------------------
\begin{frame}
\begin{itemize}
    
    \item Lipschitz continuity condition :
    \begin{itemize}
        \item determin if the transformation is stable to additive noise
        \item there exists C $>$ 0 such that for all x and x' such that : 
        \begin{equation}
             \left\|\mathbf \Phi x' - \Phi x \right\| \leq C \left\|\mathbf x' - x \right\|
        \end{equation}
    \end{itemize}
    
    \item Lipschitz continuous to deformations :
    \begin{itemize}
        \item Lipschitz continuity relative to deformations is obtained if there exists C $>$ 0 such that for all $\tau$ and x:
        \begin{equation}
             \left\|\mathbf \Phi x_{\tau} - \Phi x \right\| \leq C \left\|\mathbf x \right\| \sup_{u} \lvert \nabla (\tau(u)) \rvert
        \end{equation}
    \end{itemize}


	
\end{itemize}
\end{frame}
%------------------------------------------------

%------------------------------------------------
\begin{frame}
\begin{itemize}
	\item  wavelet transform vs Fourier transform :
	\begin{itemize}
	    \item wavelets are stable to deformations (+) but they are translation covariant (-)
	    \item Fourier sinusoidal waves are not stable to deformations (-) but they are translation invariant (+)
	\end{itemize}

\end{itemize}
\end{frame}
%------------------------------------------------

%------------------------------------------------
\begin{frame}
\begin{itemize}
    
    \item Notation
    \begin{itemize}
        \item G : group of rotations \textbf{r} of angles $2k\pi/K$ for $0 \leq k \leq K$
        \item Two-dimensional directional wavelets are obtained by rotating a single band-pass filter $\psi$ by $r \in G$ and dilating it by $2^{j}$ for $j \in \mathbb{Z}$.
        \begin{equation}
            \psi_{\lambda}(u) = 2^{-2j} \psi(2^{-j}r^{-1}u) \; with \; \lambda = 2^{-j}r
        \end{equation}
    \end{itemize}

	\item A wavelet transform commutes with translations, and is therefore not translation invariant.
	\begin{itemize}
	    \item Solution : introduce nonlinearity : Q 
	    \begin{equation}
	        Qx = M(x*\psi_{\lambda})
	    \end{equation}
	    \item The operator M should commute with the action of any diffeomorphism
	\end{itemize}

\end{itemize}
\end{frame}
%------------------------------------------------

%----------------------------------------------------------------------------------------

\end{document} 